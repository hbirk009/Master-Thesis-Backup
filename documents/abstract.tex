\documentclass[main.tex]{subfiles}

\begin{document}

\normalsize
 \vspace*{30pt}
 \begin{center}

\textbf{Abstract}\\ 

\hspace{10pt}

\end{center}
\normalsize
The ProtonCT-project is a collaboration established at the University of Bergen and several institutions worldwide to create a prototype pCT-scanner. The pCT-scanner would be used to improve dosimetry plans for proton therapy, by measuring the relative stopping power of the protons, instead of calculating it using a traditional CT-scan.

The prototype employs a Digital Tracking Calorimeter to measure the energy of the protons. This calorimeter is distinct in its design; it comprises 43 layers of pixel sensors. The pixel detectors were developed for the ALICE project by CERN. Each layer is comprised of 12 strings of ALPIDE chips, where a string is 9 ALPIDE chips mounted on a flex cable. These layers can track the trajectory and measure the energy of protons, which can be used to measure the relative stopping power. 

The power delivery to the 43 layers of the calorimeter is still under development. Currently, a Monitoring Board is used to deliver the power to the strings and monitor their current consumption, voltage levels, and temperature. This thesis presents a control system solution for the power delivery system. This solution employs system design methodology to create a robust and modular design that can configure and monitor the power delivery to the layers. IPbus, a reliable communication protocol developed at CERN, is used to transmit data between the components of the power delivery system. Several levels of API abstraction have been designed to interface with IPbus and perform various functions, such as configuring the Monitoring Board and monitoring the measurements from the board.

Database systems have been developed to store data from the configuration and monitoring processes. A web interface has been deployed that gives the user an overview of the monitoring data. Lastly, GUIs have been developed to serve as the interface between the user, the databases and the Monitoring Board.

Tests and verification were performed of the power control system, and the results showed that the control system fulfilled the timing requirements for this project. Finally, a discussion is made on the structure of the control system. The discussion covers the control system and how it can be reused and expanded upon to cover the other systems of the pCT-project.





\end{document}