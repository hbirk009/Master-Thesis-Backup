\documentclass[main.tex]{subfiles}

\begin{document}

\section{Outlook and Conclusion}
\textit{This chapter discusses the work that has been done in this thesis and the result of said work. A discussion is made on the future of the power delivery system and how it is related to the other systems of the \gls{pct}-project.}

\subsection{Summary}

The objective of this thesis was to create a control system for the power delivery of the \gls{pct}-project. The task involved creating a configuration, and monitoring system for the power delivery, A Human to Machine Interface to the system in the form of \gls{gui}s, and a database system for storing configuration and monitoring data. Additionally, the system needed to be structured with clear levels of abstraction to make it easier to expand upon in the future.

It was decided to create several levels of abstraction of the \gls{api} using Python classes. These \gls{api}s were interfaced with the MB Hub and the Monitoring Board, which comprises the Power Control system. A high-level \gls{api} for the configuration process was created. This \gls{api} incorporates a ramp-up algorithm to turn on the ALPIDE-sensors while configuring the Monitoring Board safely. MongoDB was chosen as the database used for the configuration system, and a Python class was designed to interface with the database and format the configuration sets. A \gls{gui} was created that allowed the user to load configuration sets from the database and create them interactively using the \gls{gui}. The timing of the configuration process was calculated and measured to be satisfactory for the project.

A monitoring \gls{api} was created to poll data and insert it into the database. InfluxDB was chosen for storing data with timestamps, and Grafana was used to display the data to the user. A filtering class was created that allowed for filtering data points based on OSIsoft's exception filter. A simple \gls{gui} was also created to set up the monitoring process.

A simulation was created to verify the early designs of the control software, and further tests were performed on the Power Control System prototype to verify the functionality communication link between the control software and the Monitoring Board. However, the tests showed that the reliability of the Power control system needed to be revised, and a prototype redesign was warranted.

Lastly, a discussion was made on the modularity of the control software and how it may be reused and expanded upon in the future.

\subsection{Outlook}
The control software presented in this thesis is functional and documented, which serves as a foundation for further development in the future. The system verification must still be performed with actual hardware; ideally, a testbench in the future should be created for the control software, allowing quick software verification. Handling errors from the microcontroller must be designed and implemented, ideally with a logging database, and this system must be integrated with configuration and monitoring processes.

A discussion has been made on the reusability of the control software in the context of other \gls{pct}-systems. Using a similar structure and code for the other systems is promising due to the generic control software. In particular, the low-level \gls{api}s must be redefined for the system interface. In particular, software relating to the databases is not directly tied to PCS and therefore has the potential to be reused in other configuration and monitoring systems.

There is still room for optimization for the control system, but tests on the system have shown it is more than fast enough for this project. The optimization may become more relevant when we include readout and cooling in the process.

\subsection{Conclusion}

This thesis has presented a control software that can be used for the power delivery system of the \gls{pct}-project. This control software consists of several levels of \gls{api}s, \gls{gui}s, databases for configuration and monitoring, web interfaces, and test functions for the Power Control System. Tests concluded that the control software fulfilled the requirements of the \gls{pct}-project.


\end{document}