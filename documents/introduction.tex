\documentclass[main.tex]{subfiles}

\begin{document}

\section{Introduction}

\subsection{Background}

Cancer is the second most prevailing cause of death in the world, behind cardiovascular diseases. It was reported 14 million new cases of various cancers worldwide in 2019 \cite{porcine_2021} and as life expectancy increase in the world, it is also expected for cancer rates to increase. There are several treatments for cancers, among them is \gls{rt}, a relatively new treatment developed in the early 1900s using x-rays to irradiate harmful tumours. It is an effective and non-invasive procedure, capable of removing and preventing the growth of tumours. The main challenge of \gls{rt} is delivering a lethal dose to the tumour without harming healthy tissue in the patient. A \gls{ct}-scan is often done before radiation therapy as part of calculating the dose distribution. A \gls{ct}-scan is done by sending an x-ray source through the patient, and detectors behind the patient measures the energy of the exiting photons. The density of the tissue is proportional to the absorption of the X-ray beams, meaning photons going through bone structures will lose most of their energy, while photons going through soft or no tissue will retain most of the energy. This allows us to characterize the energy lost using the Hounsfield scale, a scale specifically designed for use in \gls{ct}-scans. measuring the Hounsfield units for the proton beam allows us to calculate the required dose with higher certainty so that less radiation goes into healthy tissue. But even with these techniques, the patient will still receive a large radiation dose in nearby regions of the tumour. Using \gls{rt} in high-risk areas, such as the brain or heart can therefore lead to severe damage to the organs.  \par

Particle therapy is a new treatment that started developing in the mid 1900s as an alternative to \gls{rt}. Particle therapy was sought as an alternative because particles allows for  concentrating the dose distribution, which avoids damaging healthy tissue. Particles release very little of their energy while moving through a medium, it is not until the particle completely stops it effectively releases all its energy. This phenomenon is known as "Bremsstrahlung" and it allows us to deliver high doses of energy to a concentrated area while having relatively little entry dose and effectively zero exit dose. The moment when the particle releases all its energy is known as the "Bragg-Peak" and we can calibrate this "peak", targeting only the tumour inside the body. Knowing the \gls{rsp} of the medium the particles are sent through is essential to control the Bragg-Peak. A traditional \gls{ct}-scanner can be used to calculate the \gls{rsp}, but it is not perfect and it will lead to higher uncertainty in the measurements. The stochastic nature of particles' energy loss also leads to higher uncertainties which makes it even more crucial to have precise measurements when performing particle treatment. A solution to reduce the uncertainty is to perform the CT-scan with the same particles used for the therapy, directly measuring the \gls{rsp} of the particle in the tissue.



\subsection{Bergen PCT project}
Today the most common type of particle therapy is proton therapy; there are currently over 33 proton therapy centers in Europe. In Norway it is planned to open two proton therapy centers by the end of 2024, one in Oslo and one in Bergen. The Bergen pCT-project is a collaboration with the University of Bergen, several other universities and companies to build a \gls{pct}-scanner to be used in particle therapy. The \gls{pct}-scanner is realized with a \gls{dtc} to measure the energy levels and trajectory of the protons exiting a patient exposed to a proton beam. The \gls{dtc} is developed using \gls{alpide}-chips that was originally used for the \gls{its} in the \acrshort{alice} experiment at \acrshort{cern}. The \gls{dtc} is made out of 43 layers, and a layer is made out of 12 staves with 9 \gls{alpide}-chips each. the protons enter the front of the detector and their energy is measured by the chips as they travel through each layer. 2 of the 43 layers is designated tracking layers, calculating the angle the protons exit the patient from. It is necessary to track the angle of the protons due to effects such as coulomb-scattering which can affect both the energy and trajectory of the protons. A \gls{dtc} is traditionally made with two tracking layers in the front and two layers in the rear of the \gls{dtc}, but the \gls{pct}-project only has the rear trackers. Such a \textit{single-sided} system has advantages over a \textit{double-sided} system, such as reducing cost and complexity of the system. 

The power delivery to the \gls{dtc} is one of the ongoing areas of development in the \gls{pct}-project. The current plan is to deliver the power to the staves through a \gls{tc}, which will transfer power and perform readout of the staves. A separate \gls{mb} is responsible for delivering the power to the \gls{tc} and monitor temperature and current of the strings. Data transmission between the client and hardware is performed using IPbus, an established ethernet-based communication protocol often used in particle physics experiment. The \gls{mb} must be able to deliver enough current, and monitor temperature and current draw of the strings of \gls{alpide}-chips. The current plan is to have a power supply for each layer of the \gls{dtc} and a monitor board for each layer delivering the power to the \gls{tc}, while monitoring the strings. \par

The \gls{mb} uses a microcontroller to directly monitor the temperature and current draw of the layer, this allows us to quickly turn off staves if temperature or current exceeds safe ranges. For this purpose, the microcontroller has several configurable registers for threshold values and readings of analog current, digital current and temperature. The \gls{mb} requires a configuration system for powering the staves in a safe manner without damaging the equipment. A client-side monitoring system is also needed, that is able to give the user accurate information of the staves during a run. A monitoring system is essential for troubleshooting and debugging. The configuration and monitoring systems must also be able to quickly and reliably receive and transmit data between the user and the monitor boards, slow configuration time could have a big impact if repeated tests are performed.

\subsection{Objective of this thesis}

The objective of this thesis is to design and build the configuration and monitoring system for the \gls{pct} power delivery to the staves. Many aspects of the systems must be assessed and considered to ensure they are suitable for the \gls{pct}-project. Both systems must be able to communicate with the microcontroller on the monitoring board and as such, an \acrshort{fpga}-solution is used with IPbus to receive and transmit data between client software and the microcontrollers. 

This thesis is of a similar nature as a previous master thesis written by Jonas H. Rasmussen, that focused on the configuration of the readout of the \gls{alpide}-chips. Therefore it is natural to use the same tools used in that thesis for communicating with IPbus and creating GUIs. Python is chosen as the programming language of choice, as it is a popular language with many functionalities, and has support for uHAL, the end user API for writing and reading to registers in IPbus. In addition, Python also has pyQT5 libraries which enables for easy GUI integration with Python programs.

Two databases are required, one time-series database for monitoring the data from the staves and one database for storing configuration data used in runs. MongoDB was already in consideration for the configuration task before this thesis began. MongoDB is designed to store vast amounts of data, which is a desirable quality when storing many configuration settings. For monitoring, a time-series database is required, there are many databases to choose from. The Influx database was chosen for this project for several reasons. It is open source, popular and is in continuous development. One of the main appeals to InfluxDB is its integration with Grafana, an open source monitoring solution for databases, which has an intuitive, easy to use GUI with integrated connectivity to InfluxDB. The Grafana interface is highly customizable, allowing for displaying many graphs and statistics at once.

These databases will also require custom API functions to connect with the IPbus module. The monitoring database only needs to be able to insert values into the database and therefore need minimal integration with IPbus. Using Python libraries, it is possible to manipulate the data points before inserting them into the database, making it possible to apply custom filtering to data.  The configuration database is a part of the ramp-up algorithm and the configuration GUI, meaning it will require several custom functions for loading and saving configuration sets in the database and start the ramp-up algorithm. The configuration GUI itself must also be made to allow users with little or no knowledge of the system to create and load configuration sets for the \gls{mb}.

A general API must also be made to communicate with the \gls{fpga} that functions as an intermediary between the microcontroller and software. This API ideally will be generic, to make it reusable in different systems. A general requirement for the software in this work is to be modular and object oriented. Large projects naturally gets more complex as more features is added and \gls{oop} naturally leads to an intuitive and natural structure of the code. This is obviously important to ensure that the work in this thesis can be continued and improved upon in future work without wasting development time. Setting up a generic and functional model for the configuration and monitoring of the power supply also helps us to set up a similar system that can be used for the readout-system in the future.

The requirements for this control system is as follows:

\begin{itemize}
    \item Configure all relevant registers in the microcontrollers on the \gls{mb} for every layer.
    \item Store configuration data in a database allowing for easy loading and saving configuration sets.
    \item A GUI that allows a user with little knowledge of the system to load and create configuration sets.
    \item A powering-on algorithm that turns on the staves one-by-one, leading to a slow ramp-up of power usage, to avoid creating a large power spike.
    \item Periodically read current draw and temperature from each layer and store the data in a time series database.
    \item Display the monitoring data from each layer in intuitive and easy to understand graphs.
    \item A low and high level API to communicate with the IPbus module on the \acrshort{fpga} and the microcontroller on the \gls{mb}.
    \item Generic, modular \acrshort{oop} code that is easy to change and modify, so the system can be expanded upon and reused for similar projects in the future.
    
\end{itemize}






\end{document}