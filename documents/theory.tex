\documentclass[main.tex]{subfiles}

\begin{document}

\section{Radiotherapy and Proton Therapy}

\subsection{Radiotherapy}

Cancer is an umbrella term for many diseases involving out-of-control cell growth in the body, the most common types being breast, lung, and prostate cancer\cite{cancerData}. \gls{rt} as treatment started in the early 1900s, using X-rays to kill malignant tumours, and has become a standard procedure to remove or prevent the growth of malignant tumours. External \gls{rt} uses an external radiation source to deliver the dose. In contrast, internal \gls{rt} injects a radioactive element as a liquid into the body or near the malignant tumour. This thesis will focus on external \gls{rt} methods.
 
 \subsubsection{History and General Use}
 
 \gls{rt} was historically performed by using an external X-ray source and sending the radiation into the patient to treat growths or lesions from diseases such as lupus or skin cancer. \autoref{fig: rt_treatment_history} shows a patient being treated with X-rays before the dangers of radiation therapy were known. Today, the procedure is performed in a similar manner, but a better understanding of radiation behaviour and dosage planning enables us to target cancers inside the patient and better treat life-threatening diseases. Imaging techniques such as \gls{ct}-scans allow us to measure the radiodensity of the tissue in a patient, which can be used to create a dosage plan for \gls{rt}. Not only does this allow us to target tumours deep inside the body, but it also allows for higher accuracy dose distribution, which minimizes the amount of radiation absorbed by healthy tissue.
 
  \begin{figure}[!htpb]
    \centering
    \includegraphics[width=8cm ]{images/x_ray_treatment_history.jpg}
    \caption{Image showing a patient being treated for tuberculosis in 1910, using X-rays.\cite{rt_history}}
    \label{fig: rt_treatment_history}
\end{figure}
 
 Compared to chemotherapy and surgery, \gls{rt} has the advantage of being a non-invasive procedure and can target specific areas in the body. This allows us to remove malignant tumours more effectively and is often used alongside surgery or chemotherapy to ensure the entire tumour is removed. The downside of \gls{rt} is that when the malignant tumour is irradiated, so is the healthy tissue around the tumour, which damages it and can lead to more health issues. Using \gls{rt} in tumours near critical organs, such as the heart or brain, is difficult and must be carefully calibrated to avoid damaging healthy tissue. Using \gls{rt} in areas such as the head and neck has proven to cause many long-term side effects in patients, such as cognitive impairment and neurological damage. In rare instances, a new, local cancer may emerge\cite{headRTData}. Therefore, reducing the delivered dose to the tissue is crucial.
 
 External delivery of radiation dose to a tumour is difficult because the healthy tissue in front of the tumour absorbs large amounts of the dose, which can damage the tissue. Alternative methods must therefore be applied to treat internal tumours in a patient. A solution is using \gls{imrt}, a method where multiple, weaker radiation sources are focused on the tumour. This effectively spreads the high entry dose throughout the body, sparing the healthy tissue while still giving the tumour a high dose. A study was done to compare the effectiveness of \gls{imrt} in comparison to the use of \gls{hdr} brachytherapy \cite{imrtVShdr}. \autoref{fig: imrtvshdr} shows dose distribution of \gls{imrt} versus \gls{hdr}.

 \begin{figure}[!htpb]
    \centering
    \includegraphics[width=8cm ]{images/imrt vs hdr.jpg}
    \caption{Image showing dose distribution targeting a tumour(red circle) using HDR(a) and IMRT(b)\cite{imrtVShdr}}
    \label{fig: imrtvshdr}
\end{figure}
\FloatBarrier 

\autoref{fig: imrtvshdr} shows the dose is spread throughout the body with \gls{imrt}, compared to \gls{hdr}, where the dose is centred near the tumour. The study concluded that \gls{imrt} provided a substantial decrease in dose delivered to the organs-at-risk near the tumour, compared to \gls{hdr}\cite{imrtVShdr}. \gls{imrt} is a viable treatment for treating tumours, but a large dose is still delivered to healthy tissue using this method. Furthermore, \gls{imrt} may not be a viable treatment option if the tumour is close to critical organs, such as the brain or heart.

\subsubsection{CT-scan}
\notinmain{kanskje nevn ein plass her om ka som er forventet av ein ideell proton terapi og korleis ein egentlig er i virkeligheten}
As mentioned before, \gls{ct}-scans are performed before \gls{rt}-treatment to calculate the body's radiodensity, which is the body's attenuation to radiation. These measurements are then used in dosage planning. It can also be used as a general imaging machine of a patient's insides, which can be used for diagnostics. A \gls{ct}-scan uses X-rays and computer technology to image a "slice" of the patient's body; many of these slice pictures are taken and put together to form a slice-by-slice model of the entire body of the patient.

A \gls{ct}-scan is normally realized with an X-ray tube that rotates around the patient that sends radiation through a slice of the patient. Sensors on the other side of the machine detect the photons as they leave the patient. \autoref{fig: ct_image} shows a typical setup when performing a \gls{ct}-scan.

 \begin{figure}[!htpb]
    \centering
    \includegraphics[width=8cm ]{images/CTXRAYScan.png}
    \caption{Image showing a typical setup for a CT-scanner\cite{CTimage}}
    \label{fig: ct_image}
\end{figure}
\FloatBarrier 

The photons from the X-rays lose their energy as they move through a medium, in this case, the patient. The energy loss from the photon is proportional to the density of the tissue. The detectors on the back of the patient measure the remaining energy of the photons exiting the patient and can, in turn, calculate the radiodensity of the path the photon travelled through the body.

Radiodensity measurements used in \gls{ct}-scans are measured in the Hounsfield-scale, which uses \gls{hu}. \gls{hu}, also called CT-unit, is a linear transformation of the absorption coefficient of the X-ray beam. 0 \gls{hu} is arbitrarily set to be the energy lost as the X-ray travels through water, and -1000 \gls{hu} is the energy lost when travelling through air. Using this unit, we measure the relative energy absorption of different tissues in body and use computer technology to reconstruct this information into a picture of the patient's insides. \autoref{fig: ct_lung_image} shows one slice from a \gls{ct}-scan performed on the patient's lungs.


 \begin{figure}[!htpb]
    \centering
    \includegraphics[width=12cm ]{images/High-resolution_computed_tomograph_of_a_normal_thorax,_axial_plane_(38).jpg}
    \caption{Image slice of a patient's lungs. Image down right of the image shows what slice of the lung is displayed. By Mikael Häggström, used with permission.}
    \label{fig: ct_lung_image}
\end{figure}
\FloatBarrier 


\subsection{Proton Therapy}

Particle therapy is a medical treatment similar to \gls{rt}. However, high-energy particles are used instead of X-rays to irradiate the body. The most common one today is protons. The idea of particle therapy came about in the mid-1900s; scientists discovered that, unlike photons, high-energy particles do not deposit their energy gradually as they move through a medium; instead, they sharply deposit most of it in one area. It was therefore theorized that proton therapy could be used as an alternative to \gls{rt} to limit dose delivery to healthy tissue. 

\autoref{fig: bragg_peak} shows this sharp dose distribution of particles compared to the dose distribution of photons in water.

 \begin{figure}[!htpb]
    \centering
    \includegraphics[width=10cm ]{images/bragg_peak.jpeg}
    \caption{Graph displaying the relative dose given in water by photons, protons and carbon-12, as function of distance. The bragg peak of proton and carbon-12 is shown to the right.\cite{bragg_peak_image}}
    \label{fig: bragg_peak}
\end{figure}
\FloatBarrier 

From the figure, photons reach their max dose delivery early and slowly decrease as it moves in the water. On the other hand, protons have a relatively low dose delivery until it reaches the Bragg peak, where almost all energy is deposited. The reason protons deposit their energy is due to a phenomenon called "Bremsstrahlung", where the particle loses most of its energy as it slows down in the medium. To put it in another way, the energy loss of a particle in a medium is inversely proportional to the particle's velocity, creating the Bragg peak.

The effectiveness of proton therapy can be shown by comparing the dose delivered with conventional radiotherapy. \autoref{fig: imrt_vs_photon} compares the dose distribution of \gls{imrt} treatment and proton treatment of lung cancer in a patient.

\begin{figure}[!htpb]
    \centering
    \includegraphics[width=12cm ]{images/proton_vs_imrt.png}
    \caption{Image showing the dose distribution of IMRT treatment(left) and proton therapy(right) of a tumour in the lungs\cite{protonimage}. The upper image shows top-down image of the dose distribution, the lower image shows the same distribution from the side. }
    \label{fig: imrt_vs_photon}
\end{figure}
\FloatBarrier

One can observe from the figure that \gls{imrt} irradiates a large part of healthy tissue outside the tumour, as the entry and exit dose is high as predicted by \autoref{fig: bragg_peak}. On the other hand, the proton treatment has a lower entry dose, and most radiation is limited to the area around the tumour. Proton therapy allows us to irradiate the tumour without harming the healthy tissue in the body.

Proton therapy has the potential as a treatment option, but it comes with challenges that must be tackled before beginning treatment. There are political and economic challenges with proton therapy, including sizing the facility to house the particle accelerator. However, this thesis will focus on the physics challenge of proton therapy. Proton particles naturally behave very differently from massless photons. This means that photon dose and range calculations are insufficient for protons.\cite{proton_challenges}

Range uncertainties can significantly impact proton therapy compared to \gls{rt}. \autoref{fig: proton_uncertainty} compares the dose distribution between radiation therapy and proton therapy and displays how range uncertainty can affect the dose distribution. The photons have a high entry dose, and the dose delivered is reduced as photons travel deeper into the tissue. The proton dose is delivered by tuning the Bragg Peak of the beam to cover a small part of the tumour. After delivering a high dose, the beam is tuned again, so the Bragg Peak covers another part of the tumour. This process is repeated until the entire tumour has been given a high dose. This creates a Spread Out Bragg Peak (SOBP) distribution, as shown in \autoref{fig: proton_uncertainty}.

\begin{figure}[!htpb]
    \centering
    \includegraphics[width=12cm ]{images/projected_dose_graph.pdf}
    \caption{Image showing the effect of range uncertainties in photons(blue) and protons(orange)\cite{proton_challenges}.}
    \label{fig: proton_uncertainty}
\end{figure}
\FloatBarrier


 The dotted line is the projected dose to be delivered to the tumour, and the full line is the actual dose delivered due to range uncertainties in the \gls{rsp} measurement. From the graph of the photons, we can see that, although there is a difference between the projected dose delivered and the actual dose, it does not impact the dose distribution in a significant way. This is because the dose delivered to the tumour is approximately the same, regardless of the range uncertainties.
 
 The deviation between the projected and actual dose is more detrimental for the protons. Because the dose delivery of the protons is much sharper, an undershoot in the dose delivery results in parts of the tumour not being covered by the Bragg Peak. Therefore almost no dose is delivered to it. Likewise, suppose there was an overshoot in the dose delivery. In that case, healthy tissue could potentially enter the Bragg Peak, resulting in high irradiation of the tissue, which could cause harm to the patient. This illustrates that range uncertainty has a much more significant impact on proton therapy than radiation therapy.

The range calculations for proton therapy are traditionally done by performing a \gls{ct}-scan of the patient, and the \acrlong{hu}s from the scan is converted to \gls{rsp}. \gls{rsp} is a measurement of the stopping power of a particle moving through water, which can be used to estimate particle dose delivery. The \gls{hu} measurement has uncertainty to it. Although the uncertainty is not necessarily increased as one converts from \gls{hu} to \gls{rsp}, it will have a more significant impact on particle therapy as opposed to \gls{rt}\cite{proton_challenges}.

A solution to avoid uncertainty in the measurement is to measure the \gls{rsp} of the tissue directly instead of converting it from \gls{hu}. This can be done by performing a \gls{ct}-scan with the same particle used for treatment, in this case, a \gls{pct}-scan. By measuring \gls{rsp} directly with a \gls{pct}-scan, it is expected to achieve a higher range precision than the conventional conversion method used with a normal \gls{ct}-scan. A study was performed to test the accuracy of a \gls{pct}-scan versus a \gls{ct}-scan. A fresh post-mortem porcine structure was used to compare the \gls{rsp} measurement between a direct \gls{pct}-measurement, and the \gls{rsp} calculated from the \gls{ct}-scan\cite{porcine_2021}.

First, a measurement of the accuracy of the \gls{pct} was done by performing a \gls{pct}-scan of eight vials containing different tissue, shown in \autoref{fig: rsp_test}. The \gls{rsp} value of these different tissues was measured beforehand, with a multi-layered ionization chamber, and the \gls{rsp} values from that experiment are compared with the \gls{rsp} from the \gls{pct}-scan.

\begin{figure}[!htpb]
    \centering
    \includegraphics[width=8cm ]{images/porcine_phantom.jpg}
    \caption{Image of the RSP test done on several vials containing porcine tissue\cite{porcine_2021}}
    \label{fig: rsp_test}
\end{figure}
\FloatBarrier

The test revealed that there was up to 1\% difference between the \gls{rsp} measured with \gls{pct} and the results from the ionization chamber. The range uncertainty from converting from \gls{hu} to \gls{rsp} has been shown to be up to 3.5\%\cite{Paganetti_2012}. This suggests that \gls{pct} yields higher accuracy measurements than a \gls{ct}-scan. 


Another test compared the calculated \gls{rsp} data from a \gls{ct}-scan, with the measured \gls{rsp} from a \gls{pct}-scan of the pig's head. The result is shown in \autoref{fig: porcine_pct_vs_ct}.

\begin{figure}[!ht]
    \centering
    \includegraphics[scale=2.5]{images/porcine_comparison.jpg}
    \caption{Image showcasing difference in measured RSP and calculated RSP of the porcine structure. (a) Measured. (b) Calculated. (c) difference between measured and calculated RSP \cite{porcine_2021}.}
    \label{fig: porcine_pct_vs_ct}
\end{figure}
\FloatBarrier

The data reveals a relatively small difference between the calculated \gls{rsp} and the measured \gls{rsp} in soft tissue, with an average difference of only 2\% between them. Structures such as teeth and bone show larger discrepancies between measured and calculated RSP; there is up to a 30\% difference between them. The results from this test show that there is a significant advantage to measuring the \gls{rsp} value directly using \gls{pct} in comparison to calculating the \gls{rsp} values through converting Hounsfield units. This suggests scans of bone structures benefit more from using \gls{pct}. From this, we can conclude that \gls{pct} shows potential to outperform traditional \gls{ct} when measuring \gls{rsp} in a patient.    

\end{document}